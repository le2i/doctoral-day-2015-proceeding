%
% ACPR template packages
\usepackage{acpr}
\usepackage{times}
\usepackage{epsf,graphicx,subfig}
\usepackage{epsfig}
\usepackage{epstopdf}
\usepackage{amsmath}
\usepackage{amssymb}
\usepackage{booktabs}
\usepackage{array, multirow}
\usepackage{colortbl,bigdelim}
\usepackage{arydshln}
\usepackage[utf8]{inputenc} 
%  The following command loads a graphics package to include images
%  in the document. It may be necessary to specify a DVI driver option,
%  e.g., [dvips], but that may be inappropriate for some LaTeX
%  installations.
\usepackage[]{graphicx}

% In order to include files without having a clear page using \include*,
% the newclude package is required
\usepackage{newclude}

% Required for acronyms
% use \acresetall to reset the acroyms counter
% macros=True, allows for calling \myTriger rather than \ac{myTriger}
\usepackage[single=true, macros=true, xspace=true]{acro}

% Use biblatex to manage the referencing
%
\usepackage[style=ieee, backend=biber, backref=true]{biblatex}

% Drawing stuff
\usepackage{tikz}
\usetikzlibrary{fit, patterns, shapes, backgrounds, positioning}

% Include other packages here, before hyperref.

% If you comment hyperref and then uncomment it, you should delete
% egpaper.aux before re-running latex.  (Or just hit 'q' on the first latex
% run, let it finish, and you should be clear).
\usepackage[%pagebackref=true,
            breaklinks=true,
            letterpaper=true,
            colorlinks,
            bookmarks=false,
           ]{hyperref}

% Clever cross referencing. Using cleverref, instead of writting
% figure~\ref{...} or equation~\ref{...}, only \cref{...} is required.
% The package interprates the references and introduces the figure, fig.,
% equation, eq., etc keywords. \Cref forces first letter capital.
% >> WARNING: This package needs to be loaded after hyperref, math packages,
%             etc. if used.
%             Cleveref is recomended to load late
\usepackage{cleveref}

% To create random text use lipsum
\usepackage{lipsum}

% SI units
\usepackage{siunitx}
\DeclareSIUnit\px{px}

%%% Local Variables:
%%% mode: plain-tex
%%% TeX-master: t
%%% End:

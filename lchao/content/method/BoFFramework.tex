\tikzstyle{block} = [rectangle, draw, fill=gray!20, text = black,
    text width=6em, text centered, rounded corners, minimum height=4em , minimum width = 6em]
    % \tikzstyle{line} = [draw, -latex']
  \tikzstyle{myarrow}=[->, thick]
    \tikzstyle{line}=[-, thick]
    \tikzstyle{block2} = [rectangle, draw, fill=white!20,
    text width=6em, text centered, rounded corners, minimum height=4em, minimum width = 6em]
    \tikzstyle{block3} = [rectangle, draw, fill=gray!20, text = black,
    text width=7em, text centered, rounded corners, minimum height=4em , minimum width = 7em]
\def\blockdist{1}
\def\edgedist{1.5}
  %%%% The Framework Sparse Coding 

\begin{figure*}
 \begin{center}
   \begin{tikzpicture}[node distance = 1cm,scale=0.6, every node/.style={scale=0.6}]
%(FEx.east|- FEx.south)
    \node [block2] (input) {Training image};
    %\node [block, right of = input, node distance = 2.8cm](Seg){Segmentation}; 
    \node [block, right of=input,node distance = 2.8cm](De){Denoising};
    \node [block, right of=De,node distance = 2.8cm](FEx){Feature extraction};
    \path (FEx.east)+(+0.8,0) node (g) {};
    
    %%% Sparse Coding Block
    \node [block3, right of=g,node distance = 1.7cm](DL){Dictionary learning /k-means};
    \node [block3, below of=DL,node distance = 2.5cm](PR){Projection};
    \begin{pgfonlayer}{background}
      \path (DL.west |- DL.north)+(-0.4,-0.1+\blockdist) node (a) {};
      \path (PR.east |- PR.south)+(+0.4,-0.7) node (b) {};          
      \path[fill=gray!10,rounded corners, draw=gray!20, dashed] (a) rectangle (b);
    \end{pgfonlayer}
\path (DL.west |- DL.north)+(+1.2,-0.5+\blockdist) node (SP) {\textbf{Bag of Features}};
\path (PR.east |- PR.south)+(-1.3,-0.4+\blockdist) node (c){};
\path (PR.east)+(-3.15,0) node (d) {};

%%% Testing 
\node [block, below of=FEx, node distance = 2.5cm](FE2){Feature extraction};
\node [block, below of=De, node distance = 2.5cm](De2){Denoising};
% \node [block, below of=Seg, node distance = 2.5cm](Seg2){Segmentation}; 
\node [block2, below of=input, node distance = 2.5cm](TestImg){Testing image};

%%% 
\node [block, right of=PR, node distance = 3.6cm](Pool){Visual words histogram};
\path (Pool.east) + (0.3,0) node (f){}; 
\path (Pool.east) + (0.2,-0.1) node (f1){}; 

%%% Classification
\node [block, right of = Pool, node distance = 3.5cm] (Pre){Prediction}; 
    \node [block, above of = Pre, node distance = 2.5cm] (Learn){Learning}; 
    \begin{pgfonlayer}{background}
      \path (Learn.west |- Learn.north)+(-0.4,-0.1+\blockdist) node (h) {};
    \path (Pre.east |- Pre.south)+(+0.4,-0.7) node (i) {};          
    \path[fill=gray!10,rounded corners, draw=gray!20, dashed] (h) rectangle (i);
\end{pgfonlayer}
\path (Learn.west |- Learn.north)+(+1.1,-0.5+\blockdist) node (Clas) {\textbf{Classification}};
\path (Pre.east |- Pre.south)+(-1.3,-0.4+\blockdist) node (j){};
\path (f1.north)+(0, 2.5) node (k) {};
\path (Pre.east) + (1.2,0) node (k1) {P(..)}; 

    % Draw edges
    \draw [line] (input) -- (De) -- (FEx); 
    \draw [myarrow] (FEx)-- (DL);
    \draw [myarrow] (DL) -- (PR) ; 
    \draw [line] (TestImg) -- (De2) -- (FE2); 
    \draw [myarrow] (FE2) -- (PR) ;
    \draw [line] (PR) -- (Pool); 
    \draw [myarrow] (Pool) -- (Pre); 
    \draw [line] (f1.north) -- + (0,2.5)(k.south); 
    \draw [myarrow] (k.south)+ (0,0.1)  -- (Learn.west); 
    \draw [myarrow] (Pre) -- (k1);

    \end{tikzpicture}
    \end{center}
    

\caption{Bag of features framework} 
\label{fig:BoF-framework}

\end{figure*}
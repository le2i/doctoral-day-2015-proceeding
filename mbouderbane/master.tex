\documentclass[10pt,twocolumn,letterpaper]{article}

%% Latex documents that need direct input
% ACPR template packages                                                                                                               
\usepackage{acpr}
\usepackage{times}
\usepackage{epsf,graphicx,subfig}
\usepackage{epsfig}
\usepackage{epstopdf}
\usepackage{amsmath}
\usepackage{amssymb}
\usepackage{booktabs}
\usepackage{array, multirow}
\usepackage{colortbl,bigdelim}
\usepackage{arydshln}
\usepackage[utf8]{inputenc}
\usepackage[]{graphicx}
\usepackage{newclude}
\usepackage[single=true, macros=true, xspace=true]{acro}
\usepackage[style=ieee, backend=biber, backref=true]{biblatex}

% Mathematics extra symols and commands
\usepackage{amssymb, amsmath}
\usepackage{pifont,amsfonts} % import fonts for tick and x-mark
% define the extra symbols
\newcommand{\cmarkgLarge}{\text{\large \color{green!60!black!80}\ding{51}}}
\newcommand{\cmarkrLarge}{\text{\large \color{red!60!black!80}\ding{51}}}
\newcommand{\xmarkLarge}{\text{\large \color{red!60!black!80}\ding{55}}}
\newcommand{\cmark}{\text{\color{green!60!black!80}\ding{51}}}
\newcommand{\xmark}{\text{\color{red!60!black!80}\ding{55}}}

%% In order to draw some graphs
\usepackage{tikz,xifthen}
\usepackage{tikz-qtree}
\usetikzlibrary{decorations.pathmorphing} % noisy shapes
\usetikzlibrary{fit}                                            % fitting shapes to coordinates
\usetikzlibrary{backgrounds}                                    % drawing the background after the foreground
\usetikzlibrary{shapes,arrows,shadows}
\usetikzlibrary{calc,decorations.pathreplacing,decorations.markings,positioning}
\usetikzlibrary{snakes,decorations.text,shapes,patterns}
\usetikzlibrary{snakes}
\usetikzlibrary{decorations}
\usetikzlibrary{decorations.text}
\usetikzlibrary{decorations.markings}
\usetikzlibrary{shapes}
\usetikzlibrary{patterns}
\usepackage{pgfplots}

%%----- To generate stand onle tikz legends
% argument #1: any options
\newenvironment{customlegend}[1][]{%
    \begingroup
    % inits/clears the lists (which might be populated from previous
    % axes):
    \csname pgfplots@init@cleared@structures\endcsname
    \pgfplotsset{#1}%
}{%
    % draws the legend:
    \csname pgfplots@createlegend\endcsname
    \endgroup
}%
% makes \addlegendimage available (typically only available within an
% axis environment):
\def\addlegendimage{\csname pgfplots@addlegendimage\endcsname}
\pgfkeys{/pgfplots/number in legend/.style={%
        /pgfplots/legend image code/.code={%
            \node at (0.295,-0.0225){#1};
        },%
    },
}
%%---- end tikz legends

\usepackage[%pagebackref=true,
            breaklinks=true,
            letterpaper=true,   
            colorlinks,                                                                                                                
            bookmarks=false,                                                                                                           
           ]{hyperref} 

% Clever cross referencing. Using cleverref, instead of writting 
% figure~\ref{...} or equation~\ref{...}, only \cref{...} is required.
% The package interprates the references and introduces the figure, fig.,
% equation, eq., etc keywords. \Cref forces first letter capital. 
% >> WARNING: This package needs to be loaded after hyperref, math packages,
%             etc. if used.
%             Cleveref is recomended to load late
%\usepackage{hyperref}
\usepackage{cleveref}

% SI units
\usepackage{siunitx}
% Define the money way to write
\sisetup{
  group-four-digits = true,
  group-separator = {,}
}
\DeclareSIUnit\px{px}

% Nice package with citeauthor
%\usepackage{natbib}        % contains the latex packages
\title{Overview of ghost correction for HDR video stream generation}

\author{Mustapha Bouderbane, Pierre-Jean Lapray, Julien Dubois, Barth\'el\'emy Heyrman, Dominique Ginhac\\
Universit\'e de Bourgogne Franche-Comt\'e\\
Facult\'e Mirande, 21000 Dijon\\}
             % contains the Title and Autor info
\input{latex/filesystem/fileSetup.tex}      % contains package and variables init.
%% Acronym definition example using glossaries package
%% \usepackage{acro} is required
%% 
%% For a powerful usage of the acro package look at http://tex.stackexchange.com/questions/135975/how-to-define-an-acronym-by-using-other-acronym-and-print-the-abbreviations-toge

\DeclareAcronym{cnn}{
  short = CNN,
  long = Condensed Nearest Neighbour
}

\DeclareAcronym{nn}{
  short = NN,
  long = Nearest Neighbour
}

\DeclareAcronym{oss}{
  short = OSS,
  long = One-Sided Selection
}

\DeclareAcronym{smote}{
  short = SMOTE,
  long = Synthetic Minority Over-sampling TEchnique
}

\DeclareAcronym{enn}{
  short = ENN,
  long = Edited Nearest Neighbour
}

\DeclareAcronym{svm}{
  short = SVM,
  long = Support Vector Machines
}

\DeclareAcronym{cad}{
  short = CAD, 
  long = Computer-Aided Diagnosis
}

\DeclareAcronym{clbp}{
  short = CLBP,
  long = Completed Local Binary Pattern
}
\DeclareAcronym{lbp}{
  short = LBP,
  long = Local Binary Pattern
}
\DeclareAcronym{rf}{
  short = RF,
  long = Random Forests
}

\DeclareAcronym{ncr}{
  short = NCR,
  long = Neighborhood Cleaning Rule
}

\DeclareAcronym{se}{
  short = SE,
  long = Sensitivity
}

\DeclareAcronym{sp}{
  short = SP,
  long =  Specificity
}

\DeclareAcronym{wracc}{
  short = WRacc,
  long = Weighted Relative Accuracy
}

\DeclareAcronym{fpr}{
  short = FPR,
  long = False Positive Rate
}

\DeclareAcronym{nm}{
  short = NM,
  long = NearMiss
}

\DeclareAcronym{nm1}{
  short = NM1,
  long = NearMiss-1
}

\DeclareAcronym{nm2}{
  short = NM2,
  long = NearMiss-2
}

\DeclareAcronym{nm3}{
  short = NM3,
  long = NearMiss-3
}

\DeclareAcronym{os}{
  short = OS,
  long = Over-Sampling
}

\DeclareAcronym{ros}{
  short = ROS,
  long = Random Over-Sampling
}

\DeclareAcronym{us}{
  short = US,
  long = Under-Sampling
}

\DeclareAcronym{rus}{
  short = RUS,
  long = Random Under-Sampling
}

\DeclareAcronym{cus}{
  short = CUS,
  long = Clustering Under-Sampling
}

\DeclareAcronym{bd}{
  short = BD,
  long = Barrel Deformation
}

\DeclareAcronym{rdgm}{
  short = RDGM,
  long = Random Deformation using Gaussian Motion 
}

\DeclareAcronym{tl}{
  short = TL,
  long = Tomek Link 
}      % contains the acronims 

\acprfinalcopy % *** Uncomment this line for the final submission
\def\acprPaperID{438} % *** Enter the acpr Paper ID here
\def\httilde{\mbox{\tt\raisebox{-.5ex}{\symbol{126}}}}
% Pages are numbered in submission mode, and unnumbered in camera-ready
\pagestyle{empty}


%% Select inputing only one part of the document
%\includeonly{content/intro/intro}   % the file wihtout .tex
%\includeonly{content/other/other_content}
 
\addbibresource{./content/literature_review.bib}

\begin{document} 
\maketitle 

% \begin{abstract}
% This paper addresses the problem of automatic classification of \ac{sdoct} data for automatic identification of patients with \ac{dme} versus normal subjects.
% Our method is based on \ac{lbp} features to describe the texture of \ac{oct} images and we compare different \ac{lbp} features extraction approaches to compute a single signature for the whole \ac{oct} volume.
% Experimental results with two datasets of respectively 32 and 30 \ac{oct} volumes show that
% regardless of using low or high level representations, features derived from \ac{lbp} texture have highly discriminative power.% for the task on hand.
% Moreover, the experiments show that the proposed method achieves better classification performances than other recent published works.
% %\keywords{\acl{dme}, \acl{oct}, \acs{dme}, \acs{oct}, \ac{lbp}.}
% \end{abstract}

%% Incldue the content without .tex extension
\acresetall  % reset the acronyms from the abstract
%% include the figures path relative to the master file
\graphicspath{ {./content/intro/figures/} }

\section{Introduction}
\label{sec:intro}  % \label{} allows reference to this section

Malignant melanoma is the deadliest type of skin cancer, accounting for the vast majority of skin cancer deaths~\cite{CancerFactsFigures2014}. 
According to latest reports, melanoma causes over 20,000 deaths annually in Europe~\cite{forsea2012melanoma}. 
In 2014, the American Cancer Society also reported that the number of new diagnosed cases is 76,100 with 9710 estimated deaths~\cite{CancerFactsFigures2014}. 
%Nevertheless, early diagnosis play a key role since that melanoma being the most treatable kind of cancer.
Nevertheless, melanoma is the most treatable kind of cancer if diagnosed early. 

The clinical diagnosis of early stage melanoma is commonly based on the ``ABCDE'' rule~\cite{abbasi2004early}, defined as Asymmetry, irregular Borders, variegated Colours, Diameters greater than \SI{6}{\milli \metre} and Evolving stages over time. 
In addition, the clinical diagnosis of melanoma is performed through visual inspection and deep analysis of the lesion, using clinical imaging techniques such as dermoscopic imaging. 
However, these inspections and analysis are not easy tasks due to challenges such as similarity of the different lesion types (dysplastic and melanoma) and the necessity to perform patient follow-up over years.
Therefore, the research communities have dedicated their efforts to develop computerized lesion analysis algorithms for classification of melanoma lesions. 
However, akin to other medical applications, the percentage of melanoma cases in comparison with benign and dysplatic cases is far less. 
This problem is frequently referred as ``class imbalanced'' problem~\cite{prati2009data} and has been encountered in multiple areas such as telecommunication managements, bioinformatics, fraud detection, and medical diagnosis. 
Imbalanced data substantially compromise the learning process since most of the standard machine learning algorithms expect balanced class distribution or an equal misclassification cost~\cite{he2009learning}.

Medical data are prone to such drawbacks due to the fact that the portion of diseased samples or patients is far lower than healthy cases.
Furthermore, the detection and classification of minority malignant cases are highly essential so that the \ac{se} of developed algorithms needs to be maximized.
Consequently, the problem of imbalanced data is usually addressed by employing different techniques which do not vitiate the topology of the data.
Despite the fact that classification of malignant melanoma has been extensively studied~\cite{rastgoo2015automatic}, up to our knowledge, only two works tackled the issue implied by imbalanced dataset~\cite{barata2013two,celebi2007methodological}.
Barata~\etal generate new synthetic samples by adding a Gaussian noise with fixed parameters to the samples belonging to the minority class~\cite{barata2013two}.
Celebi~\etal over-sampled their dataset using \ac{smote}~\cite{chawla2002smote} to improve the \ac{se} of their algorithm~\cite{celebi2007methodological}.

This paper provides an insight to the specific problem of classification of imbalanced dataset for malenoma. 
To proceed, we review different techniques proposed by the machine learning community and compile a comprehensive quantitative evaluation. The rest of this paper is organized as follows: an overview of the classification framework designed to investigate data balancing techniques is presented in Sect.\,\ref{sec:mm} while these strategies are described in Sect.\,\ref{sec:met}. A quantitative evaluation is discussed in Sect.\,\ref{sec:exp-res} followed by a concluding section.


% Some stuff that emac's colegues use
%%% Local Variables: 

%%% mode: latex
%%% TeX-master: "../../master"
%%% End: 

          % the file wihtout .tex
%% include the figures path relative to the master file
\graphicspath{ {./content/method/figures/} }

\begin{figure*}
	\subfloat[][]{
	\label{fig:GridOriginal}\includegraphics[width=0.26\textwidth, height = 0.08\textheight]{OG.png}}\hfill
	\subfloat[][]{
	\label{fig:GridGaussian}\includegraphics[width=0.26\textwidth, height = 0.08\textheight]{GG3_80.png}}\hfill
	\subfloat[][]{
	\label{fig:GridBarrel}\includegraphics[width=0.26\textwidth, height = 0.08\textheight]{BG3_145.png}}
	\caption{Data space transformation: \protect\subref{fig:GridOriginal} original synthetic data, \protect\subref{fig:GridGaussian} \acs*{rdgm} deformation, \protect\subref{fig:GridBarrel} \acs*{bd} deformation.}
	\label{fig:DSOS}
\end{figure*}

\section{Balancing strategies}\label{sec:met}
Considering a binary classification problem, the class with the smallest number of samples is defined as the \textit{minority} class and its counterpart is defined as the \textit{majority} class.
The problem of data balancing corresponds to equalizing the number of samples of both the minority and majority classes. This task can be achieved in either data or feature space.

\subsection{Data space sampling}

Data space sampling is related with the generation of new synthetic samples by modifying the original data ahead of any feature extraction processes.
\Ac{os} is performed on the original dataset by generating synthetic melanoma images based on two types of deformation~\cite{rastgoo2015ensemble}. Furthermore, cubic b-spline interpolation is used with both methods to approximate non-integer points in the image.
 
\begin{description}
	\item[\Ac{rdgm}] achieved by deforming the original image by adding a random Gaussian motion $\mathcal{N}(\mu, \sigma) = (5,5)$ at each pixel compounded with a global rotation of~\SI{80}{\degree}.
	\item[\Ac{bd}] corresponds to a deformation of the original image using barrel distortion compounded with a global rotation of~\SI{145}{\degree}.
\end{description}

A synthetic example illustrating the results of these deformation is presented in Fig.\,\ref{fig:DSOS}.

% \begin{table*}
% \caption{The obtained results with different balancing techniques for color and texture features using a \acs*{rf} classifier. The first and second highest results for each feature set are highlighted in dark and lighter gray colors respectively.}
% \centering
% \resizebox{1.\textwidth}{!}{ \begin{tabular}{l cccccc		cccccc		cccccc}
\toprule
Features &  \multicolumn{6}{l}{Color}& \multicolumn{6}{l}{Texture} & \multicolumn{6}{l}{Combined}\\
  \cmidrule(r){2-7}  \cmidrule(r){8-13}  \cmidrule(r){14-19}  
		   & \multicolumn{2}{c}{$C_{1}$}& \multicolumn{2}{c}{$C_{2}$}& \multicolumn{2}{c}{$C_{1,2}$}& \multicolumn{2}{c}{$T_{1}$} &  \multicolumn{2}{c}{$T_{2}$} & \multicolumn{2}{c}{$T_{1,2}$}& \multicolumn{2}{c}{$T_{1},C_{1,2}$}& \multicolumn{2}{c}{$T_{2},C_{1,2}$}& \multicolumn{2}{c}{$T_{1,2},C_{1,2}$}\\ 
  \cmidrule(r){2-3}  \cmidrule(r){4-5} \cmidrule(r){6-7} \cmidrule(r){8-9} \cmidrule(r){10-11} \cmidrule(r){12-13} \cmidrule(r){14-15} \cmidrule(r){16-17} \cmidrule(r){18-19} 
  Balancing techniques & \makebox[0.5cm][l]{\acs*{se}}& \makebox[0.5cm][l]{\acs*{sp}} & \makebox[0.5cm][l]{\acs*{se}} & \makebox[0.5cm][l]{\acs*{sp}} & \makebox[0.5cm][l]{\acs*{se}} &\makebox[0.5cm][l]{\acs*{sp}} &\makebox[0.5cm][l]{\acs*{se}} &\makebox[0.5cm][l]{\acs*{sp}} &\makebox[0.5cm][l]{\acs*{se}} & \makebox[0.5cm][l]{\acs*{sp}} &\makebox[0.5cm][l]{\acs*{se}} &\makebox[0.5cm][l]{\acs*{sp}} & \makebox[0.5cm][l]{\acs*{se}} &\makebox[0.5cm][l]{\acs*{sp}} &\makebox[0.5cm][l]{\acs*{se}} &\makebox[0.5cm][l]{\acs*{sp}}& \makebox[0.5cm][l]{\acs*{se}} &\makebox[0.5cm][l]{\acs*{sp}}  \\ \midrule
\multicolumn{1}{c}{IB} & 52.50 & 89.58 & 75.00 & 88.75 & 71.25& 87.50& 38.75 & 91.67 & 60.00 & 96.25 & 66.25 & 93.75 &73.75 & 89.58&71.25 & 89.58 & 71.25& 92.50\\
\midrule \midrule
\multicolumn{1}{c}{\acs*{os}} &\cellcolor[gray]{0.6}93.75 &\cellcolor[gray]{0.6}66.67 &80.00 & 86.25& 82.50 & 87.08  & 43.75 &83.75 &72.50& 90.00 & 70.00& 91.67  &77.50 & 87.08 &81.25 &88.33 &78.75 &88.33  \\
\midrule \midrule
\multicolumn{1}{c}{\acs*{ros}} &55.00 & 80.83& 80.00 & 84.17& 72.50 &85.42 &42.50 & 82.08 &60.00 & 89.17 &66.25 &87.92&75.00&85.42&73.75&86.25&73.75 &85.83\\
\multicolumn{1}{c}{\acs*{smote}} & 60.00 & 82.50 & 78.75 & 84.58 & 75.00 & 70.00 & 56.25 & 74.17 & 61.25 & 87.50 & 84.17 &87.08 & 78.75& 85.00 &73.75 & 84.58 &73.75&85.00 \\ 
\midrule
\multicolumn{1}{c}{\ac{rus}} & 72.50 & 72.92 & 86.25 & 80.00 & 78.75 &80.00 & 67.50 & 53.33 &76.25 &76.25  &85.00 &78.75 &\cellcolor[gray]{0.6}91.25 & \cellcolor[gray]{0.6}75.00 & 85.00 & 78.75 &\cellcolor[gray]{0.6}92.50 &\cellcolor[gray]{0.6}78.33\\
\multicolumn{1}{c}{\acs*{tl}} & 51.25 & 86.25 & 76.25 & 87.92&67.50 & 88.33  & 37.50 & 87.92 & 65.00 &90.42 & 68.75 & 91.67 & 73.75 & 88.75 &63.75 & 90.00 & 72.50 & 91.25\\
\multicolumn{1}{c}{\acs*{cus}} & 81.25 & 67.92 & 80.00 & 84.58&\cellcolor[gray]{0.8} 86.25 & \cellcolor[gray]{0.8}80.42 & 56.25 & 65.83 & 70.00 & 77.50 & 85.00 & 77.08 & 83.75 & 81.25 & 80.00 & 84.17 & 83.75 & 82.92\\
\multicolumn{1}{c}{\acs*{nm1}} & 67.50 & 72.08 & 86.25 & 79.17& 85.00 & 82.50 & 72.50 & 43.75 & 80.00 & 62.50 &\cellcolor[gray]{0.6} 87.50 &\cellcolor[gray]{0.6} 66.67 & 85.00 & 82.08 & \cellcolor[gray]{0.8}86.25 &\cellcolor[gray]{0.8}80.42 & 87.50 & 80.83\\
\multicolumn{1}{c}{\acs*{nm2}} & 70.00 & 72.92 & 86.25 & 81.25 & 85.00 & 82.92 & \cellcolor[gray]{0.8}76.25 &\cellcolor[gray]{0.8} 48.75& \cellcolor[gray]{0.6}86.25 &\cellcolor[gray]{0.6} 40.83 & \cellcolor[gray]{0.8}86.25 &\cellcolor[gray]{0.8} 51.25& \cellcolor[gray]{0.8}87.50 & \cellcolor[gray]{0.8}82.08 &\cellcolor[gray]{0.6}92.50 &\cellcolor[gray]{0.6}77.50& \cellcolor[gray]{0.8}91.25 &\cellcolor[gray]{0.8}81.67\\
\multicolumn{1}{c}{\acs*{nm3}} & \cellcolor[gray]{0.8}82.50 & \cellcolor[gray]{0.8}75.00 &\cellcolor[gray]{0.8} 87.50 &\cellcolor[gray]{0.8} 80.83 & 85.00 & 80.42 &73.75 & 55.83 & 72.50 & 82.50 & 82.50 & 80.42 & 83.75 & 81.25 & 85.00 & 80.00 & 86.25 & 80.42\\
\multicolumn{1}{c}{\ac{ncr}} & {\color{blue}66.25} & {\color{blue}76.67} & \cellcolor[gray]{0.6}{\color{blue} 87.50} &\cellcolor[gray]{0.6}{\color{blue} 81.25} &{\color{blue}85.00} &{\color{blue} 82.08} & {\color{blue}67.50} & {\color{blue}67.92} & {\color{blue}75.00} & {\color{blue}85.83} & {\color{blue} 82.50} & {\color{blue} 83.33} & {\color{blue} 86.25} & {\color{blue} 81.67} & {\color{blue}82.50} &{\color{blue} 85.00} & {\color{blue}83.75} &{\color{blue} 85.42}\\
\midrule
\multicolumn{1}{c}{\acs*{smote} + \acs*{enn}} & 76.25 & 73.33 & 85.00 & 81.25 & 85.00 & 82.08 &\cellcolor[gray]{0.6} 81.25 &\cellcolor[gray]{0.6} 56.25 & 76.25 & 82.08 & 80.00 & 79.58 & 86.25 & 81.25 & 83.75 & 82.50 & 78.75 & 82.92\\
\multicolumn{1}{c}{\acs*{smote} + \acs*{tl}} & 75.00 & 73.75 & 83.75 & 82.50 & \cellcolor[gray]{0.6}87.50 &\cellcolor[gray]{0.6}80.83 & 72.50 & 59.17 & \cellcolor[gray]{0.8}77.50 &\cellcolor[gray]{0.8} 82.08 & 78.75 & 78.75 & 85.00 & 82.08 & 77.50 & 82.92 & 88.75 & 82.50\\
\bottomrule
\end{tabular}
}
% \label{tab:tab1}
% \end{table*}

\begin{table*}
\caption{The obtained results with different balancing techniques for color and texture features using a \acs*{rf} classifier. The first and second highest results for each feature set are highlighted in dark and lighter gray colors, respectively.}
\centering
\resizebox{1.\textwidth}{!}{
\begin{tabular}{l cccccc		cccccc		cccccc}
\toprule
Features &  \multicolumn{6}{l}{Color}& \multicolumn{6}{l}{Texture} & \multicolumn{6}{l}{Combined}\\
  \cmidrule(r){2-7}  \cmidrule(r){8-13}  \cmidrule(r){14-19}  
		   & \multicolumn{2}{c}{$C_{1}$}& \multicolumn{2}{c}{$C_{2}$}& \multicolumn{2}{c}{$C_{1,2}$}& \multicolumn{2}{c}{$T_{1}$} &  \multicolumn{2}{c}{$T_{2}$} & \multicolumn{2}{c}{$T_{1,2}$}& \multicolumn{2}{c}{$T_{1},C_{1,2}$}& \multicolumn{2}{c}{$T_{2},C_{1,2}$}& \multicolumn{2}{c}{$T_{1,2},C_{1,2}$}\\ 
  \cmidrule(r){2-3}  \cmidrule(r){4-5} \cmidrule(r){6-7} \cmidrule(r){8-9} \cmidrule(r){10-11} \cmidrule(r){12-13} \cmidrule(r){14-15} \cmidrule(r){16-17} \cmidrule(r){18-19} 
  Balancing techniques & \makebox[0.5cm][l]{\acs*{se}}& \makebox[0.5cm][l]{\acs*{sp}} & \makebox[0.5cm][l]{\acs*{se}} & \makebox[0.5cm][l]{\acs*{sp}} & \makebox[0.5cm][l]{\acs*{se}} &\makebox[0.5cm][l]{\acs*{sp}} &\makebox[0.5cm][l]{\acs*{se}} &\makebox[0.5cm][l]{\acs*{sp}} &\makebox[0.5cm][l]{\acs*{se}} & \makebox[0.5cm][l]{\acs*{sp}} &\makebox[0.5cm][l]{\acs*{se}} &\makebox[0.5cm][l]{\acs*{sp}} & \makebox[0.5cm][l]{\acs*{se}} &\makebox[0.5cm][l]{\acs*{sp}} &\makebox[0.5cm][l]{\acs*{se}} &\makebox[0.5cm][l]{\acs*{sp}}& \makebox[0.5cm][l]{\acs*{se}} &\makebox[0.5cm][l]{\acs*{sp}}  \\ \midrule
\multicolumn{1}{c}{IB} & 52.50 & 89.58 & 75.00 & 88.75 & 71.25& 87.50& 38.75 & 91.67 & 60.00 & 96.25 & 66.25 & 93.75 &73.75 & 89.58&71.25 & 89.58 & 71.25& 92.50\\
\midrule \midrule
\multicolumn{1}{c}{\acs*{os}} &\cellcolor[gray]{0.6}93.75 &\cellcolor[gray]{0.6}66.67 &80.00 & 86.25& 82.50 & 87.08  & 43.75 &83.75 &72.50& 90.00 & 70.00& 91.67  &77.50 & 87.08 &81.25 &88.33 &78.75 &88.33  \\
\midrule \midrule
\multicolumn{1}{c}{\acs*{ros}} &55.00 & 80.83& 80.00 & 84.17& 72.50 &85.42 &42.50 & 82.08 &60.00 & 89.17 &66.25 &87.92&75.00&85.42&73.75&86.25&73.75 &85.83\\
\multicolumn{1}{c}{\acs*{smote}} & 60.00 & 82.50 & 78.75 & 84.58 & 75.00 & 70.00 & 56.25 & 74.17 & 61.25 & 87.50 & 84.17 &87.08 & 78.75& 85.00 &73.75 & 84.58 &73.75&85.00 \\ 
\hdashline \noalign{\vskip 3pt}
\multicolumn{1}{c}{\acs*{rus}} & 72.50 & 72.92 & 86.25 & 80.00 & 78.75 &80.00 & 67.50 & 53.33 &76.25 &76.25  &85.00 &78.75 &\cellcolor[gray]{0.6}91.25 & \cellcolor[gray]{0.6}75.00 & 85.00 & 78.75 &\cellcolor[gray]{0.6}92.50 &\cellcolor[gray]{0.6}78.33\\
\multicolumn{1}{c}{\acs*{tl}} & 51.25 & 86.25 & 76.25 & 87.92&67.50 & 88.33  & 37.50 & 87.92 & 65.00 &90.42 & 68.75 & 91.67 & 73.75 & 88.75 &63.75 & 90.00 & 72.50 & 91.25\\
\multicolumn{1}{c}{\acs*{cus}} & 81.25 & 67.92 & 80.00 & 84.58&\cellcolor[gray]{0.8} 86.25 & \cellcolor[gray]{0.8}80.42 & 56.25 & 65.83 & 70.00 & 77.50 & 85.00 & 77.08 & 83.75 & 81.25 & 80.00 & 84.17 & 83.75 & 82.92\\
\multicolumn{1}{c}{\acs*{nm1}} & 67.50 & 72.08 & 86.25 & 79.17& 85.00 & 82.50 & 72.50 & 43.75 & 80.00 & 62.50 &\cellcolor[gray]{0.6} 87.50 &\cellcolor[gray]{0.6} 66.67 & 85.00 & 82.08 & \cellcolor[gray]{0.8}86.25 &\cellcolor[gray]{0.8}80.42 & 87.50 & 80.83\\
\multicolumn{1}{c}{\acs*{nm2}} & 70.00 & 72.92 & 86.25 & 81.25 & 85.00 & 82.92 & \cellcolor[gray]{0.8}76.25 &\cellcolor[gray]{0.8} 48.75& \cellcolor[gray]{0.6}86.25 &\cellcolor[gray]{0.6} 40.83 & \cellcolor[gray]{0.8}86.25 &\cellcolor[gray]{0.8} 51.25& \cellcolor[gray]{0.8}87.50 & \cellcolor[gray]{0.8}82.08 &\cellcolor[gray]{0.6}92.50 &\cellcolor[gray]{0.6}77.50& \cellcolor[gray]{0.8}91.25 &\cellcolor[gray]{0.8}81.67\\
\multicolumn{1}{c}{\acs*{nm3}} & \cellcolor[gray]{0.8}82.50 & \cellcolor[gray]{0.8}75.00 &\cellcolor[gray]{0.8} 87.50 &\cellcolor[gray]{0.8} 80.83 & 85.00 & 80.42 &73.75 & 55.83 & 72.50 & 82.50 & 82.50 & 80.42 & 83.75 & 81.25 & 85.00 & 80.00 & 86.25 & 80.42\\
\multicolumn{1}{c}{\acs*{ncr}} & {\color{blue}66.25} & {\color{blue}76.67} & \cellcolor[gray]{0.6}{\color{blue} 87.50} &\cellcolor[gray]{0.6}{\color{blue} 81.25} &{\color{blue}85.00} &{\color{blue} 82.08} & {\color{blue}67.50} & {\color{blue}67.92} & {\color{blue}75.00} & {\color{blue}85.83} & {\color{blue} 82.50} & {\color{blue} 83.33} & {\color{blue} 86.25} & {\color{blue} 81.67} & {\color{blue}82.50} &{\color{blue} 85.00} & {\color{blue}83.75} &{\color{blue} 85.42}\\
\hdashline \noalign{\vskip 3pt}
\multicolumn{1}{c}{\acs*{smote} + \acs*{enn}} & 76.25 & 73.33 & 85.00 & 81.25 & 85.00 & 82.08 &\cellcolor[gray]{0.6} 81.25 &\cellcolor[gray]{0.6} 56.25 & 76.25 & 82.08 & 80.00 & 79.58 & 86.25 & 81.25 & 83.75 & 82.50 & 78.75 & 82.92\\
\multicolumn{1}{c}{\acs*{smote} + \acs*{tl}} & 75.00 & 73.75 & 83.75 & 82.50 & \cellcolor[gray]{0.6}87.50 &\cellcolor[gray]{0.6}80.83 & 72.50 & 59.17 & \cellcolor[gray]{0.8}77.50 &\cellcolor[gray]{0.8} 82.08 & 78.75 & 78.75 & 85.00 & 82.08 & 77.50 & 82.92 & 88.75 & 82.50\\
\bottomrule
\end{tabular}
}
\label{tab:tab1}
\end{table*}


\subsection{Feature space sampling}

%Considering a binary classification problem, the class with the smallest number of samples is defined as the \textit{minority} class and its counterpart is defined as the \textit{majority} class.
%
%The problem of data balancing corresponds to equalizing the number of samples of both the minority and majority classes.
%Subsequently, 
Considering the problem of imbalanced, \ac{us} is performed such that the number of samples of the majority class is reduced to be equal to the number of samples of the minority class.
The following methods are considered to perform such balancing.

\begin{description}
  \item[\Ac{rus}] is performed by randomly selecting without replacement a subset of samples from the majority class such that the number of samples is then equal in both minority and majority classes.
  \item[\Ac{tl}] can be used to under-sample the majority class of the original dataset~\cite{tomek1976two}.
Let define a pair of \ac{nn} samples $(x_i, x_j)$ such that their associated class label $y_i \neq y_j$.
The pair $(x_i, x_j)$ is defined as a \ac{tl} if, by relaxing the class label differentiation constraint, there is no other sample $x_k$ defined as the \ac{nn} of either $x_i$ or $x_j$.
\Ac{us} is performed by removing the samples belonging to the majority class and forming a \ac{tl}.
It can be noted that this \ac{us} strategy does not enforce a strict balancing between the majority and the minority classes.
  \item[\Ac{cus}] refers to the use of a $k$-means to cluster the feature space such that $k$ is set to be equal to the number of samples composing the minority class.
Hence, the centroids of theOAse clusters define the new samples of the majority class. 
  \item[\Ac{nm}] offers three different methods to under-sample the majority class~\cite{mani2003knn}.
In \ac{nm1}, samples from the majority class are selected such that for each sample, the average distance to the $k$ \ac{nn} samples from the minority class is minimum.
\ac{nm2} diverges from \ac{nm1} by considering the $k$ farthest neighbours samples from the minority class.
In \ac{nm3}, a subset $M$ containing samples from the majority class is generated by finding the $m$ \ac{nn} from each sample of the minority class.
Then, samples from the subset $M$ are selected such that for each sample, the average distance to the $k$ \ac{nn} samples from the minority class is maximum.
In our experiment, $k$ and $m$ are fixed to 3.
  \item[\Ac{ncr}] consists of applying two rules depending on the class of each sample~\cite{laurikkala2001improving}.
Let define $x_i$ as a sample of the dataset with its associated class label $y_i$.
Let define $y_m$ as the class of the majority vote of the $k$ \ac{nn} of the sample $x_i$.
If $y_i$ corresponds to the majority class and $y_i \neq y_m$, $x_i$ is rejected from the final subset.
If $y_i$ corresponds to the minority class and and $y_i \neq y_m$, then the $k$ \ac{nn} are rejected from the final subset.
\end{description}

In the contrary, the data balancing can be performed by \ac{os} in which the new samples belonging to the minority class are generated aiming at equalizing the number of samples in both classes.
Two different methods are considered.

\begin{description}
  \item[\Ac{ros}] is performed by randomly replicating the samples of the minority class such that the number of samples is equal in both minority and majority classes.
  \item[\Ac{smote}] is a method to generate synthetic samples in the feature space~\cite{chawla2002smote}.
Let define $x_i$ as a sample belonging to the minority class.
Let define $x_{nn}$ as a randomly selected sample from the $k$ \ac{nn} of $x_i$.
Therefore, a new sample $x_j$ is generated such that $x_j = x_i + \sigma \left( x_{nn} - x_i \right)$, where $\sigma$ is a random number in the interval $\left[0,1\right]$.
\end{description}

Subsequently, \ac{os} methods can be combined with \ac{us} methods to clean the subset created.
In that regard, two different combinations are tested.

\begin{description}
  \item[\ac{smote} + \ac{tl}] are combined to clean the samples created using \ac{smote}~\cite{batista2003balancing}.
\ac{smote} over-sampling can lead to overfitting which can be avoided by removing the \ac{tl} from both majority and minority classes~\cite{prati2009data}.
  \item[\ac{smote} + \ac{enn}] are combined for the same aforementioned reason~\cite{batista2004study}.
\end{description}

%%% Local Variables: 
%%% mode: latex
%%% TeX-master: "../../master"
%%% End: 
 
%\input{content/experiment/experiment}
%\input{content/results/results}

% \section{Conclusions}\label{sec:con}
% The work presented here addresses the automatic classification of \ac{sdoct} data to identify subjects with \ac{dme} versus normal.
% Based on the reported results, the low level volume 3D features and high level 2D features using patches achieve the most desirable results in the experimental setup presented here.
% The comparison against different datasets and methodologies, highlights that:
% regardless of using low or high level representations, volume signatures derived from \ac{lbp} texture show high discriminative power for distinguishing \ac{dme} vs normal volumes.

Most digital cameras use low dynamic range image sensors, these LDR sensors can capture only a limited luminance dynamic range of the scene~\cite{hang2014}, to about two orders of magnitude (about 256 to 1024 levels). However, the dynamic range of real-world scenes varies over several orders of magnitude (10.000 levels). To overcome this limitation, several methods exist for creating high dynamic range (HDR) image (expensive method uses dedicated HDR image sensor and low-cost solutions using a conventional LDR image sensor). Large number of low-cost solutions applies a temporal exposure bracketing. The HDR image may be constructed with a HDR standard method (an additional step called tone mapping is required to display the HDR image on conventional system), or by fusing LDR images in different exposures time directly, providing HDR-like~\cite{Gelfand2010} images which can be handled directly by LDR image monitors.

Temporal exposure bracketing solution is used for static scenes but it cannot be applied directly for dynamic scenes or HDR videos since camera or object motion in bracketed exposures creates artifacts called ghost~\cite{Sen2012}, in HDR image. There are a several techniques allowing the detection and removing ghost artifacts (Variance based ghost detection, Entropy based ghost detection, Bitmap based ghost detection, Graph-Cuts based ghost detection ...)~\cite{Srikantha2012}, nevertheless most of these methods are expensive in calculating time and they cannot be considered for real-time implementations.

The originality and the final goal of our work are to upgrade our current smart camera allowing HDR video stream generation with a sensor full-resolution ($1280 \times 1024$) at 60 fps~\cite{lapray2014}. The HDR stream is performed using exposure bracketing techniques (obtained with conventional LDR image sensor) combined with a tone mapping algorithm. In this paper, we propose an overview of the different methods to correct ghost artifacts which are available in the state of art. The
selection of algorithms is done concerning our final goal which is real-time hardware implementation of the ghost detection and removing phases.

{\small
\printbibliography
}

\end{document} 


\section{Related Work}\label{sec:2}

Generally, active thermography approaches use a uniform heating over the entire object-surface and the data are processed assuming the 1D model of the heat transfer. However, several works tackle this problem using local heating.
Burrows et al.\cite{Burrows2011} detect crack defects on object surface using a laser beam. The heat distribution is disrupted by the crack which highlight cracks.
In the case of non-through defects, Hammiche et al.\cite{Hammiche1996} make use of a \ac{sthm} where a probe is put into contact with the surface of the sample. Thermal contrast between the excitation and the response of the material is used to build a contrast image at each scanned point of the object.
Along the same lines, Ermert et al.\cite{Ermert1984} develop a non-contact detection approach based on a \ac{sem}. A focused and modulated electron beam excites the object surface and the infrared radiations are captured by a pyroelectric detector. However, the surface controlled is restricted.

Fiber orientation can be estimated using pulsed thermal ellipsometry technique\cite{Cielo1987}which consist in heating the part using a laser beam. The inspected part is heated by a laser beam and due to the anisotropy of the material, the observed thermal pattern become elliptical. Subsequently, The orientation of the fiber is deduced from ellipse orientation.

%%% Local Variables: 
%%% mode: latex
%%% TeX-master: "../../master"
%%% End: 

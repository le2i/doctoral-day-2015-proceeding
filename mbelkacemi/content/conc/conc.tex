\newpage

\section{Conclusion and future works}\label{sec:5}

In this paper, a system based on the \ac{sfh} approach is proposed for detecting non-through defects and fiber orientation. 
In the one hand, experiments conducted on steel and aluminum plates showed that non-through defects until \SI{2.5}{\milli \metre} are detected. 
On the other hand, a test was carried out in order to investigate the influence of the depth and the size of the defects on the defect detection.
In addition, accurate fiber orientation assessment is achieved with an experimental standard deviation of \SI{1.7}{\degree}. 
The use of the \ac{sfh} for \ac{ndt} application and the 3D digitization offers the prospect to imagine an industrial system that can give a complete solution for quality control process. 
As future works, tests will be done on different specimen to explore the influence of material properties, defect size and depth on the detection method. 
Detecting defect on complex objects is a though challenge in active thermography. 
Our technique has proven to be effective for non-through defect detection on planar object. 
The next step is to investigate the possibility of detecting defects on complex geometry objects.   
Moreover, kinetic analysis of the thermal radiation could allow to estimate the depth of the defect.

%%% Local Variables: 
%%% mode: latex
%%% TeX-master: "../../master"
%%% End: 

\documentclass[10pt,twocolumn,letterpaper]{article}

%% Latex documents that need direct input
% ACPR template packages                                                                                                               
\usepackage{acpr}
\usepackage{times}
\usepackage{epsf,graphicx,subfig}
\usepackage{epsfig}
\usepackage{epstopdf}
\usepackage{amsmath}
\usepackage{amssymb}
\usepackage{booktabs}
\usepackage{array, multirow}
\usepackage{colortbl,bigdelim}
\usepackage{arydshln}
\usepackage[utf8]{inputenc}
\usepackage[]{graphicx}
\usepackage{newclude}
\usepackage[single=true, macros=true, xspace=true]{acro}
\usepackage[style=ieee, backend=biber, backref=true]{biblatex}

% Mathematics extra symols and commands
\usepackage{amssymb, amsmath}
\usepackage{pifont,amsfonts} % import fonts for tick and x-mark
% define the extra symbols
\newcommand{\cmarkgLarge}{\text{\large \color{green!60!black!80}\ding{51}}}
\newcommand{\cmarkrLarge}{\text{\large \color{red!60!black!80}\ding{51}}}
\newcommand{\xmarkLarge}{\text{\large \color{red!60!black!80}\ding{55}}}
\newcommand{\cmark}{\text{\color{green!60!black!80}\ding{51}}}
\newcommand{\xmark}{\text{\color{red!60!black!80}\ding{55}}}

%% In order to draw some graphs
\usepackage{tikz,xifthen}
\usepackage{tikz-qtree}
\usetikzlibrary{decorations.pathmorphing} % noisy shapes
\usetikzlibrary{fit}                                            % fitting shapes to coordinates
\usetikzlibrary{backgrounds}                                    % drawing the background after the foreground
\usetikzlibrary{shapes,arrows,shadows}
\usetikzlibrary{calc,decorations.pathreplacing,decorations.markings,positioning}
\usetikzlibrary{snakes,decorations.text,shapes,patterns}
\usetikzlibrary{snakes}
\usetikzlibrary{decorations}
\usetikzlibrary{decorations.text}
\usetikzlibrary{decorations.markings}
\usetikzlibrary{shapes}
\usetikzlibrary{patterns}
\usepackage{pgfplots}

%%----- To generate stand onle tikz legends
% argument #1: any options
\newenvironment{customlegend}[1][]{%
    \begingroup
    % inits/clears the lists (which might be populated from previous
    % axes):
    \csname pgfplots@init@cleared@structures\endcsname
    \pgfplotsset{#1}%
}{%
    % draws the legend:
    \csname pgfplots@createlegend\endcsname
    \endgroup
}%
% makes \addlegendimage available (typically only available within an
% axis environment):
\def\addlegendimage{\csname pgfplots@addlegendimage\endcsname}
\pgfkeys{/pgfplots/number in legend/.style={%
        /pgfplots/legend image code/.code={%
            \node at (0.295,-0.0225){#1};
        },%
    },
}
%%---- end tikz legends

\usepackage[%pagebackref=true,
            breaklinks=true,
            letterpaper=true,   
            colorlinks,                                                                                                                
            bookmarks=false,                                                                                                           
           ]{hyperref} 

% Clever cross referencing. Using cleverref, instead of writting 
% figure~\ref{...} or equation~\ref{...}, only \cref{...} is required.
% The package interprates the references and introduces the figure, fig.,
% equation, eq., etc keywords. \Cref forces first letter capital. 
% >> WARNING: This package needs to be loaded after hyperref, math packages,
%             etc. if used.
%             Cleveref is recomended to load late
%\usepackage{hyperref}
\usepackage{cleveref}

% SI units
\usepackage{siunitx}
% Define the money way to write
\sisetup{
  group-four-digits = true,
  group-separator = {,}
}
\DeclareSIUnit\px{px}

% Nice package with citeauthor
%\usepackage{natbib}        % contains the latex packages
\title{Overview of ghost correction for HDR video stream generation}

\author{Mustapha Bouderbane, Pierre-Jean Lapray, Julien Dubois, Barth\'el\'emy Heyrman, Dominique Ginhac\\
Universit\'e de Bourgogne Franche-Comt\'e\\
Facult\'e Mirande, 21000 Dijon\\}
             % contains the Title and Autor info
\input{latex/filesystem/fileSetup.tex}      % contains package and variables init.
%% Acronym definition example using glossaries package
%% \usepackage{acro} is required
%% 
%% For a powerful usage of the acro package look at http://tex.stackexchange.com/questions/135975/how-to-define-an-acronym-by-using-other-acronym-and-print-the-abbreviations-toge

\DeclareAcronym{cnn}{
  short = CNN,
  long = Condensed Nearest Neighbour
}

\DeclareAcronym{nn}{
  short = NN,
  long = Nearest Neighbour
}

\DeclareAcronym{oss}{
  short = OSS,
  long = One-Sided Selection
}

\DeclareAcronym{smote}{
  short = SMOTE,
  long = Synthetic Minority Over-sampling TEchnique
}

\DeclareAcronym{enn}{
  short = ENN,
  long = Edited Nearest Neighbour
}

\DeclareAcronym{svm}{
  short = SVM,
  long = Support Vector Machines
}

\DeclareAcronym{cad}{
  short = CAD, 
  long = Computer-Aided Diagnosis
}

\DeclareAcronym{clbp}{
  short = CLBP,
  long = Completed Local Binary Pattern
}
\DeclareAcronym{lbp}{
  short = LBP,
  long = Local Binary Pattern
}
\DeclareAcronym{rf}{
  short = RF,
  long = Random Forests
}

\DeclareAcronym{ncr}{
  short = NCR,
  long = Neighborhood Cleaning Rule
}

\DeclareAcronym{se}{
  short = SE,
  long = Sensitivity
}

\DeclareAcronym{sp}{
  short = SP,
  long =  Specificity
}

\DeclareAcronym{wracc}{
  short = WRacc,
  long = Weighted Relative Accuracy
}

\DeclareAcronym{fpr}{
  short = FPR,
  long = False Positive Rate
}

\DeclareAcronym{nm}{
  short = NM,
  long = NearMiss
}

\DeclareAcronym{nm1}{
  short = NM1,
  long = NearMiss-1
}

\DeclareAcronym{nm2}{
  short = NM2,
  long = NearMiss-2
}

\DeclareAcronym{nm3}{
  short = NM3,
  long = NearMiss-3
}

\DeclareAcronym{os}{
  short = OS,
  long = Over-Sampling
}

\DeclareAcronym{ros}{
  short = ROS,
  long = Random Over-Sampling
}

\DeclareAcronym{us}{
  short = US,
  long = Under-Sampling
}

\DeclareAcronym{rus}{
  short = RUS,
  long = Random Under-Sampling
}

\DeclareAcronym{cus}{
  short = CUS,
  long = Clustering Under-Sampling
}

\DeclareAcronym{bd}{
  short = BD,
  long = Barrel Deformation
}

\DeclareAcronym{rdgm}{
  short = RDGM,
  long = Random Deformation using Gaussian Motion 
}

\DeclareAcronym{tl}{
  short = TL,
  long = Tomek Link 
}      % contains the acronims

\acprfinalcopy % *** Uncomment this line for the final submission
\def\acprPaperID{438} % *** Enter the acpr Paper ID here 
\def\httilde{\mbox{\tt\raisebox{-.5ex}{\symbol{126}}}}
% Pages are numbered in submission mode, and unnumbered in camera-ready                                                                
\pagestyle{empty}


%% Select inputing only one part of the document
%\includeonly{content/intro/intro}   % the file wihtout .tex
%\includeonly{content/other/other_content}

\addbibresource{./content/literature_review.bib}

\begin{document}
\maketitle

\begin{abstract}
\acresetall  % reset the acronyms from the title (if any)
As long as breast cancer remains the leading cause of cancer deaths among female population world wide, developing tools to assist radiologists during the diagnosis process is necessary.
However, most of the technologies developed in the imaging laboratories are rarely integrated in this assessing process, as they are based on information cues differing from those used by clinicians.
In order to grant \ac{cad} systems with these information cues when performing non-aided diagnosis, better segmentation strategies are needed to automatically produce accurate delineations of the breast structures.
This paper proposes a highly modular and flexible framework for segmenting breast tissues and lesions present in \ac{bus} images.
This framework relies on an optimization strategy and high-level descriptors designed analogously to the visual cues used by radiologists.
The methodology is comprehensively compared to other sixteen published methodologies developed for segmenting lesions in \ac{bus} images.
The proposed methodology achieves similar results than reported in the state-of-the-art.
%The achieved results state that the proposed methodology behaves accordingly to the state-of-the-art.
\end{abstract}

%\keywords{Breast Ultra-Sound, BI-RADS lexicon, Optimization based Segmentation, Machine-Learning based Segmentation, Graph-Cuts}

%% Incldue the content without .tex extension
\acresetall  % reset the acronyms from the abstract
\include*{content/intro/intro}          % the file wihtout .tex
\input{./content/method/figures/framework/framework_icons}
\begin{figure*}[htpb]
  \scriptsize
  \centering
  \input{./content/method/figures/framework/framework}
  % \includegraphics[width=0.9\linewidth]{method}
  \caption{Conceptual block representation of the segmentation methodology.}%\footnotemark}
    %\footnote{\cref{fig:methodTerms} illustrates the $\mathcal{S}$, $D(\cdot)$, and $V(\cdot)$ for the applied case of delineating breast structures in \ac{us} data.}
    %\footnote{(todo:add all the names of the elements in the figure)}
  \label{fig:method}
\end{figure*}
\include*{content/method/method}
\include*{content/method/applied}
%\include*{content/features/features}
\include*{content/results/results}

\section{Conclusions}
This work presents a segmentation strategy to delineate lesions in \ac{bus} images using an optimization framework that takes advantage of all the facilities available when using \ac{ml} techniques.
Despite the limitation that the final segmentation is subject to the super-pixels' boundaries, the \ac{aov} results reported here are similar to those reported by other methodologies in the literature.
A higher \ac{aov} result can be achieved by refining the delineation resulting from our proposed framework by post-processing it with an \ac{acm}. In this manner, the contour constraints could be applied to achieve a more natural delineation.

{\small 
\printbibliography                                                                                                                     }


\end{document}

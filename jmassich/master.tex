\documentclass[10pt,twocolumn,letterpaper]{article}

%% Latex documents that need direct input
%
% ACPR template packages
\usepackage{acpr}
\usepackage{times}
\usepackage{acro}
\usepackage{epsfig}
\usepackage{amsmath}
\usepackage{amssymb}
\acresetall
%  The following command loads a graphics package to include images
%  in the document. It may be necessary to specify a DVI driver option,
%  e.g., [dvips], but that may be inappropriate for some LaTeX
%  installations.
\usepackage[]{graphicx}

% In order to include files without having a clear page using \include*,
% the newclude package is required
\usepackage{newclude}

% Required for acronyms
%use \acresetall to reset the acroyms counter
%macros=True, allows for calling \myTriger rather than \ac{myTriger}
%\usepackage[single=true, macros=true, xspace=true]{acro}

% Use biblatex to manage the referencing
%
\usepackage[style=ieee, backend=biber, backref=true]{biblatex}

% Include other packages here, before hyperref.

% If you comment hyperref and then uncomment it, you should delete
% egpaper.aux before re-running latex.  (Or just hit 'q' on the first latex
% run, let it finish, and you should be clear).
\usepackage[%pagebackref=true,
            breaklinks=true,
            letterpaper=true,
            colorlinks,
            bookmarks=false,
           ]{hyperref}

% Clever cross referencing. Using cleverref, instead of writting
% figure~\ref{...} or equation~\ref{...}, only \cref{...} is required.
% The package interprates the references and introduces the figure, fig.,
% equation, eq., etc keywords. \Cref forces first letter capital.
% >> WARNING: This package needs to be loaded after hyperref, math packages,
%             etc. if used.
%             Cleveref is recomended to load late
\usepackage{cleveref}

% To create random text use lipsum
\usepackage{lipsum}
%\usepackage{translations}        % contains the latex packages
\title{Tackling the Curse of Data Imbalancing for Melanoma Classification}

\author{Mojdeh Rastgoo, Guillaume Lema\^itre, Rafael Garcia\\
Universitat de Girona\\
Campus Montilivi, Edifici P4, 17071 Girona\\
% For a paper whose authors are all at the same institution,
% omit the following lines up until the closing ``}''.
% Additional authors and addresses can be added with ``\and'',
% just like the second author.
% To save space, use either the email address or home page, not both
\and
Joan Massich, Olivier Morel, Fabrice M\'eriaudeau, Franck Marzani\\
Universit\'e de Bourgogne Franche-Comt\'e\\
12 rue de la Fonderie, 71200 Le Creusot\\
}
             % contains the Title and Autor info
%%%%%%%%%%%%%%%%%%%%%%%%%%%%%%%%%%%%%%%%%%%%%%%%%%%%%%%%%%%%% 
%>>>> uncomment following for page numbers
% \pagestyle{plain}    
%>>>> uncomment following to start page numbering at 301 
%\setcounter{page}{301} 
      % contains package and variables init.
%% Acronym definition example using glossaries package
%% \usepackage{acro} is required
%% 
%% For a powerful usage of the acro package look at http://tex.stackexchange.com/questions/135975/how-to-define-an-acronym-by-using-other-acronym-and-print-the-abbreviations-toge

\DeclareAcronym{cnn}{
  short = CNN,
  long = Condensed Nearest Neighbour
}

\DeclareAcronym{nn}{
  short = NN,
  long = Nearest Neighbour
}

\DeclareAcronym{oss}{
  short = OSS,
  long = One-Sided Selection
}

\DeclareAcronym{smote}{
  short = SMOTE,
  long = Synthetic Minority Over-sampling TEchnique
}

\DeclareAcronym{enn}{
  short = ENN,
  long = Edited Nearest Neighbour
}

\DeclareAcronym{svm}{
  short = SVM,
  long = Support Vector Machines
}

\DeclareAcronym{cad}{
  short = CAD, 
  long = Computer-Aided Diagnosis
}

\DeclareAcronym{clbp}{
  short = CLBP,
  long = Completed Local Binary Pattern
}
\DeclareAcronym{lbp}{
  short = LBP,
  long = Local Binary Pattern
}
\DeclareAcronym{rf}{
  short = RF,
  long = Random Forests
}

\DeclareAcronym{ncr}{
  short = NCR,
  long = Neighborhood Cleaning Rule
}

\DeclareAcronym{se}{
  short = SE,
  long = Sensitivity
}

\DeclareAcronym{sp}{
  short = SP,
  long =  Specificity
}

\DeclareAcronym{wracc}{
  short = WRacc,
  long = Weighted Relative Accuracy
}

\DeclareAcronym{fpr}{
  short = FPR,
  long = False Positive Rate
}

\DeclareAcronym{nm}{
  short = NM,
  long = NearMiss
}

\DeclareAcronym{nm1}{
  short = NM1,
  long = NearMiss-1
}

\DeclareAcronym{nm2}{
  short = NM2,
  long = NearMiss-2
}

\DeclareAcronym{nm3}{
  short = NM3,
  long = NearMiss-3
}

\DeclareAcronym{os}{
  short = OS,
  long = Over-Sampling
}

\DeclareAcronym{ros}{
  short = ROS,
  long = Random Over-Sampling
}

\DeclareAcronym{us}{
  short = US,
  long = Under-Sampling
}

\DeclareAcronym{rus}{
  short = RUS,
  long = Random Under-Sampling
}

\DeclareAcronym{cus}{
  short = CUS,
  long = Clustering Under-Sampling
}

\DeclareAcronym{bd}{
  short = BD,
  long = Barrel Deformation
}

\DeclareAcronym{rdgm}{
  short = RDGM,
  long = Random Deformation using Gaussian Motion 
}

\DeclareAcronym{tl}{
  short = TL,
  long = Tomek Link 
}      % contains the acronims

\acprfinalcopy % *** Uncomment this line for the final submission
\def\acprPaperID{438} % *** Enter the acpr Paper ID here 
\def\httilde{\mbox{\tt\raisebox{-.5ex}{\symbol{126}}}}
% Pages are numbered in submission mode, and unnumbered in camera-ready                                                                
\pagestyle{empty}


%% Select inputing only one part of the document
%\includeonly{content/intro/intro}   % the file wihtout .tex
%\includeonly{content/other/other_content}

\addbibresource{./content/literature_review.bib}

\begin{document}
\maketitle

\begin{abstract}
\acresetall  % reset the acronyms from the title (if any)
As long as breast cancer remains the leading cause of cancer deaths among female population world wide, developing tools to assist radiologists during the diagnosis process is necessary.
However, most of the technologies developed in the imaging laboratories are rarely integrated in this assessing process, as they are based on information cues differing from those used by clinicians.
In order to grant \ac{cad} systems with these information cues when performing non-aided diagnosis, better segmentation strategies are needed to automatically produce accurate delineations of the breast structures.
This paper proposes a highly modular and flexible framework for segmenting breast tissues and lesions present in \ac{bus} images.
This framework relies on an optimization strategy and high-level descriptors designed analogously to the visual cues used by radiologists.
The methodology is comprehensively compared to other sixteen published methodologies developed for segmenting lesions in \ac{bus} images.
The proposed methodology achieves similar results than reported in the state-of-the-art.
%The achieved results state that the proposed methodology behaves accordingly to the state-of-the-art.
\end{abstract}

%\keywords{Breast Ultra-Sound, BI-RADS lexicon, Optimization based Segmentation, Machine-Learning based Segmentation, Graph-Cuts}

%% Incldue the content without .tex extension
\acresetall  % reset the acronyms from the abstract
\include*{content/intro/intro}          % the file wihtout .tex
\include*{content/method/method}
\include*{content/method/applied}
%\include*{content/features/features}
\include*{content/results/results}

\section{Conclusions}
This work presents a segmentation strategy to delineate lesions in \ac{bus} images using an optimization framework that takes advantage of all the facilities available when using \ac{ml} techniques.
Despite the limitation that the final segmentation is subject to the super-pixels' boundaries, the \ac{aov} results reported here are similar to those reported by other methodologies in the literature.
A higher \ac{aov} result can be achieved by refining the delineation resulting from our proposed framework by post-processing it with an \ac{acm}. In this manner, the contour constraints could be applied to achieve a more natural delineation.

{\small 
\printbibliography                                                                                                                     }


\end{document}
